\documentclass{report}

\input{preamble}
\input{macros}
\input{letterfonts}

\usepackage{tikz}
\usepackage{tikz-3dplot}
\usepackage{amsmath}
\usepackage{pgfplots}
\usepackage{smartdiagram}
\usepackage{graphicx}
\usepackage{chemfig}

\usepackage{amssymb}  % For additional symbols if needed
\usesmartdiagramlibrary{additions}

\title{\Huge{Prelab}\\ \Large{John Caradonna}}
\author{\huge{Giacomo Cappelletto}}
\date{3/9/24}

\begin{document}


\maketitle
\newpage
\pdfbookmark[section]{\contentsname}{toc}
\tableofcontents
\pagebreak

\chapter{Preparation}

\section{Before}

\nt{For Labs Notes, Prepare

	\begin{enumerate}
		\item Purpose
		\item Hazards
		\item Procedure
	\end{enumerate}

	Bring to Lab
	\\
	\textbf{PreLab worksheet}
	\\
	\textbf{Long pants}
}


\section{During}

\nt{
	\begin{enumerate}
		\item Arrive early
		\item Brief lab instructor talk
		\item Do the stuff
		\item Clean up
		\item Post lab worksheet
		\item Ask TF to sign worksheet and upload to gradescope
		\item Leave
	\end{enumerate}
}

\chapter*{Lab 01 - Synthesis of a Hydrated Salt}


\begin{center}
	\textbf{Giacomo Cappelletto - 17/09/2024}	
\end{center}

\section*{Chemicals Safety}
{
	\begin{itemize}
		\item Only transfer them in proper containers (lab glassware).
		\item Do not pour solutions at or above eye level.
		\item Label any containers used for more than just immediately transferring (e.g. you don't need to label a graduated cylinder, but a beaker for storing it on the bench should be labeled).
		\item When diluting or adding to a reaction mixture, never pour a solution into a concentrated corrosive as that is more likely to splash the corrosive. Pour the corrosive into the other solution.
		\item We provide disposable gloves to protect your hands from accidental exposure in the labs. They are a temporary measure in case you get a chemical on them; they typically have a rating for how long before certain compounds pass through to your skin.
	\end{itemize}
}

\bigskip
\centerline{$\Diamond\Diamond\Diamond$} % Three diamond symbols
\bigskip

\section*{Potassium Hydroxide $KOH$}{
	\textbf{Hazards}
	\begin{enumerate}
		\item May be corrosive to metals
		\item Harmful if swallowed
		\item Causes severe skin burns and eye damage
		\item May cause respiratory irritation
	\end{enumerate}
	\textbf{Precautions}
	\begin{enumerate}
		\item Wash face, hands and any exposed skin thoroughly after handling
		\item Do not eat, drink or smoke when using this product
		\item Do not breathe dust/fume/gas/mist/vapors/spray
		\item Wear protective gloves/protective clothing/eye protection/face protection
		\item Use only outdoors or in a well-ventilated area
		\item Keep only in original container
	\end{enumerate}

}

\bigskip
\centerline{$\Diamond\Diamond\Diamond$} % Three diamond symbols
\bigskip

\section*{Ethanol (C$_2$H$_5$OH)}{
	\textbf{Hazards}
	\begin{enumerate}
		\item Highly flammable liquid and vapor
		\item Causes serious eye irritation
		\item May cause drowsiness or dizziness
		\item Can cause central nervous system depression if ingested in large quantities
		\item Harmful if inhaled in high concentrations
	\end{enumerate}
	\textbf{Precautions}
	\begin{enumerate}
		\item Keep away from heat, hot surfaces, sparks, open flames, and other ignition sources. No smoking.
		\item Use explosion-proof electrical/ventilating/lighting equipment.
		\item Wear protective gloves, protective clothing, and eye protection.
		\item Avoid breathing fumes, vapor, or spray. Use only in a well-ventilated area or under a fume hood.
		\item In case of fire, use water spray, alcohol-resistant foam, dry chemical, or carbon dioxide.
		\item Store in a tightly closed container, in a cool and well-ventilated area.
	\end{enumerate}
}

\bigskip
\centerline{$\Diamond\Diamond\Diamond$} % Three diamond symbols
\bigskip

\section*{Heating Corrosives and Releasing Gasses}{
	\textbf{Hazards}
	\begin{enumerate}
		\item Avoid boiling which could cause spillages.
		\item Caution with hot glassware.
		\item Heating corrosive substances can cause them to vaporize, leading to exposure to harmful fumes.
		\item Gases released during heating can cause respiratory irritation or even poisoning, depending on the substance.
		\item If heated in closed containers, the build-up of pressure can cause explosions or container ruptures.
		\item Certain corrosives, when heated, may react violently with other substances.
	\end{enumerate}
	\textbf{Precautions}
	\begin{enumerate}
		\item Always use appropriate glassware rated for heat when working with corrosive substances.
		\item Ensure proper ventilation, such as working under a fume hood, to avoid inhaling dangerous fumes.
		\item Never heat a closed container, especially if it contains volatile or corrosive substances.
		\item When heating, maintain a safe distance and use heat-resistant gloves and protective eyewear.
		\item Be aware of the properties of the corrosive you are heating, and avoid overheating to prevent dangerous chemical reactions.
	\end{enumerate}
}

\bigskip
\centerline{$\Diamond\Diamond\Diamond$} % Three diamond symbols
\bigskip

\section*{Guidelines on Disposable Gloves}{
	\textbf{Hazards}
	\begin{enumerate}
		\item Disposable gloves may provide insufficient protection against certain chemicals or corrosive substances.
		\item Improper glove fit can lead to tears or reduced dexterity, increasing the risk of spills or exposure.
		\item Reusing disposable gloves can lead to contamination, reducing their effectiveness.
		\item Certain gloves may cause allergic reactions, particularly those made from latex.
		\item Gloves can get punctured or damaged without visible signs, leading to exposure.
	\end{enumerate}
	\textbf{Precautions}
	\begin{enumerate}
		\item Always choose gloves appropriate for the chemicals and materials being handled (e.g., nitrile, latex, or vinyl gloves).
		\item Inspect gloves for visible damage or defects before use, such as tears, holes, or punctures.
		\item Discard gloves after each use, especially if they become contaminated or damaged.
		\item Remove gloves carefully to avoid skin contact with hazardous substances.
		\item Wash hands before and after wearing gloves to minimize contamination risks.
		\item Use gloves that fit snugly but allow for free movement to avoid accidental spills.
	\end{enumerate}
}

\bigskip
\centerline{$\Diamond\Diamond\Diamond$} % Three diamond symbols
\bigskip

\section*{Materials}

\begin{itemize}
    \item Aluminum can
    \item Steel wool or sandpaper
    \item 250 mL beakers
    \item 1.5 M potassium hydroxide, KOH
    \item Hot plate
    \item Büchner filter flask
    \item Funnel
    \item Filter paper
    \item Vacuum line
    \item Ice bath
    \item 9 M sulfuric acid, H$_2$SO$_4$
    \item Ethanol
    \item Drying oven
    \item Watch glass
\end{itemize}

\bigskip
\centerline{$\Diamond\Diamond\Diamond$} % Three diamond symbols
\bigskip

\section*{Procedure}

\begin{enumerate}
    \item Cut open a soda can and remove the top/bottom to get a rectangle of metal. \textbf{Sand off the paint and lacquer} from both sides using steel wool or sandpaper. Make sure it’s completely clean, or you’ll end up with a \textbf{sticky yellow product} that won’t work later.
    \item \textbf{Rinse} the metal with water and dry it.
    \item Cut the metal into \textbf{1 cm squares} and weigh about \textbf{1 g} of these (\textbf{record the actual mass}).
    \item Add the metal squares to a \textbf{250 mL beaker} with \textbf{40 mL of 1.5 M KOH}. Place the beaker on a \textbf{hot plate (in the fume hood)} and gently heat it, but \textbf{don’t let it boil}. Adjust the heat if bubbles start to form. This will take around \textbf{30 minutes}.
    \item While waiting, \textbf{set up a filtration apparatus} using a \textbf{500 mL Buchner flask} and funnel. \textbf{Clamp it securely}, place filter paper in the funnel, and \textbf{wet it with deionized water}. Have the TF check your setup.
    \item Once all the metal dissolves, \textbf{use gloves} to remove the beaker from the heat and let it cool slightly.
    \item While still hot, \textbf{filter the mixture}. The filtrate should be clear—if not, filter again.
    \item \textbf{Rinse the reaction beaker} with a maximum of \textbf{10 mL of deionized water} and pour it over the filter while the vacuum is running.
    \item Once filtration is done, \textbf{turn off the vacuum} and note the solids on the filter paper and the liquid in the flask.
    \item Transfer the filtrate to a \textbf{clean 250 mL beaker} and rinse the flask with another \textbf{10 mL of deionized water}, adding the rinse to the beaker.
    \item Make an \textbf{ice bath} with equal parts ice and water, record its temperature, and \textbf{cool your filtrate} in it. Don’t let the beaker tip!
    \item Add \textbf{20 mL of 9 M H$_2$SO$_4$} to the cold solution while stirring. \textbf{Note your observations}, stir for \textbf{2 minutes}, and remove from the ice bath.
    \item \textbf{Heat the mixture} in the fume hood until the \textbf{solids dissolve} and the \textbf{solution clears}.
    \item \textbf{Let the solution cool} to room temperature, then place it back in the ice bath to start \textbf{crystallization}. If nothing happens after 10 minutes, \textbf{scratch the beaker} with a glass rod to help.
    \item Once crystallization starts, leave the solution in the ice bath for an additional \textbf{10 minutes}.
    \item Reassemble and \textbf{clean your filtration apparatus}. Filter the crystals using vacuum. \textbf{Help transfer the solids} with a spatula and rinse the beaker with a little deionized water to get everything onto the filter.
    \item \textbf{Wash the crystals} with \textbf{10 mL of ethanol} (first, disconnect the vacuum). Then reconnect the vacuum to dry the crystals.
    \item Transfer the filtrate to a waste container and \textbf{pull air through the flask} for \textbf{10 minutes} to fully dry the crystals.
    \item \textbf{Place the crystals on a watch glass} and put them in the oven at \textbf{80°C for 10 minutes} to remove any remaining water and ethanol.
    \item Weigh your dried crystals by transferring them to a \textbf{pre-weighed watch glass} and \textbf{record the mass}.
    \item \textbf{Label your crystals} with your group name and section, and store them as directed for next week.
    \item Dispose of any waste in the \textbf{proper containers}, \textbf{clean up your workspace}, and return equipment. \textbf{Double-check} that the hot plate and vacuum are turned off.
    \item Have the instructor \textbf{sign off} on your \textbf{cleaned workspace}.
\end{enumerate}

\bigskip
\centerline{$\Diamond\Diamond\Diamond$} % Three diamond symbols
\bigskip

\section*{Observations and Records}

\begin{enumerate}
    \item \textbf{Mass of metal squares:} 
	\\
    \\
    \rule{15cm}{0.4pt} % Line to fill in
    
    \item \textbf{Observations of the reaction in KOH:} \\
	\\
	\rule{15cm}{0.4pt} % Line to fill in
	\\
	\\
	\rule{15cm}{0.4pt} % Line to fill in
    
    \item \textbf{Filtration observations (solids on filter paper and filtrate):} \\
	\\
	\rule{15cm}{0.4pt} % Line to fill in
	\\
	\\
	\rule{15cm}{0.4pt} % Line to fill in
    
    \item \textbf{Temperature of ice bath:} \\
	\\
	\rule{15cm}{0.4pt} % Line to fill in
    
    \item \textbf{Observations upon adding 9 M H$_2$SO$_4$:} \\
	\\
	\rule{15cm}{0.4pt} % Line to fill in
	\\
	\\
	\rule{15cm}{0.4pt} % Line to fill in
    
    \item \textbf{Observations of solution clearing after heating:} \\
	\\
	\rule{15cm}{0.4pt} % Line to fill in
	\\
	\\
	\rule{15cm}{0.4pt} % Line to fill in
    
    \item \textbf{Observations of crystallization process:} \\
	\\
	\rule{15cm}{0.4pt} % Line to fill in
	\\
	\\
	\rule{15cm}{0.4pt} % Line to fill in
    
    \item \textbf{Final mass of dried crystals:} \\
	\\
	\rule{15cm}{0.4pt} % Line to fill in
\end{enumerate}


\end{document}