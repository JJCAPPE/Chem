\documentclass{article}
\usepackage{amsmath}
\usepackage{graphicx}
\usepackage{chemfig}
\usepackage{tikz}
\usepackage{mhchem}

\title{Lab 06: Intermolecular Forces - Procedure and Safety Notes}
\author{}
\date{}

\begin{document}

\maketitle

\section*{Materials}
\begin{itemize}
    \item Digital Thermometer
    \item Beakers
    \item Test Tubes (10 mL)
    \item Test Tube Rack
    \item Ring Stand and Clamp
    \item Filter Paper Strips
    \item Masking Tape
    \item Liquids: Methanol (CH$_3$OH), 1-Propanol (CH$_3$(CH$_2$)$_2$OH), 1-Butanol (CH$_3$(CH$_2$)$_3$OH), Hexane (CH$_3$(CH$_2$)$_4$CH$_3$), Acetone (CH$_3$COCH$_3$), Deionized Water (H$_2$O), Unknown Liquid
\end{itemize}

\section*{Procedure Outline}
\begin{enumerate}
    \item \textbf{Preparation}:
    \begin{itemize}
        \item Draw Lewis structures for each liquid, indicating 3D structure with wedged and dashed bonds.
        \item Identify the intermolecular forces (IMFs) each liquid exhibits, comparing similarities and differences.
    \end{itemize}

    \item \textbf{Predict Evaporation Rates}:
    \begin{itemize}
        \item Rank compounds based on predicted evaporation rates, considering molecular size, dipole presence, and hydrogen bonding capability.
    \end{itemize}

    \item \textbf{Temperature Measurement}:
    \begin{itemize}
        \item Add 3-5 mL of the first solvent to a test tube. Place it in a fume hood.
        \item Immerse thermometer until temperature stabilizes; record as initial temperature $T_i$.
    \end{itemize}

    \item \textbf{Evaporation Process}:
    \begin{itemize}
        \item Remove the thermometer and clamp it on the stand. Begin timing evaporation as soon as it leaves the liquid.
        \item Record the minimum temperature $T_{\text{min}}$ and evaporation time $t_{\text{evap}}$ when temperature begins rising.
        \item Calculate temperature change $\Delta T = T_i - T_{\text{min}}$ and evaporation rate \( \text{Rate}_{\text{evap}} = \frac{\Delta T}{t_{\text{evap}}} \).
    \end{itemize}

    \item \textbf{Increasing Evaporation Time} (if rates are too close to distinguish):
    \begin{itemize}
        \item Wrap a filter paper strip around the thermometer tip to increase liquid contact and prolong evaporation.
    \end{itemize}

    \item \textbf{Repeat}:
    \begin{itemize}
        \item Follow steps 3-5 for each liquid.
    \end{itemize}

    \item \textbf{Cleanup}:
    \begin{itemize}
        \item Dispose of organic liquids in the designated waste container; dump water into the sink.
        \item Return equipment to storage.
    \end{itemize}
\end{enumerate}

\section*{Safety Concerns and SDS Summary}
\begin{itemize}
    \item \textbf{Methanol (CH$_3$OH)}: Highly flammable and toxic by inhalation, ingestion, and skin contact. Use in a fume hood with gloves and goggles.
    \item \textbf{1-Propanol (CH$_3$(CH$_2$)$_2$OH)}: Flammable, irritant to eyes and skin, toxic if ingested. Requires gloves, goggles, and fume hood.
    \item \textbf{1-Butanol (CH$_3$(CH$_2$)$_3$OH)}: Flammable, causes eye and skin irritation. Use standard protective equipment.
    \item \textbf{Hexane (CH$_3$(CH$_2$)$_4$CH$_3$)}: Extremely flammable, may cause dizziness or drowsiness. Proper ventilation and protective gear required.
    \item \textbf{Acetone (CH$_3$COCH$_3$)}: Highly flammable, irritant to eyes and respiratory system. Use with appropriate protective measures.
    \item \textbf{Deionized Water (H$_2$O)}: Generally safe but should be disposed of properly after use.
\end{itemize}

Each compound should be handled minimally, kept closed when not in use, and worked with in the fume hood to avoid inhaling fumes.

\section*{Lewis Structures and VSEPR Shapes}
\begin{itemize}
    \item \textbf{Methanol (CH$_3$OH)}
    \begin{center}
        % Lewis Structure for Methanol
        \chemfig{H-C(-[:30]H)(-[:-30]H)-O-H}
    \end{center}
    - \textbf{VSEPR Shape}: Tetrahedral around carbon, bent around oxygen.

    \item \textbf{1-Propanol (CH$_3$CH$_2$CH$_2$OH)}
    \begin{center}
        % Lewis Structure for 1-Propanol
        \chemfig{H-C(-[:30]H)(-[:-30]H)-C(-[:30]H)(-[:-30]H)-C(-[:30]H)(-[:-30]H)-O-H}
    \end{center}
    - \textbf{VSEPR Shape}: Tetrahedral around each carbon, bent around oxygen.

    \item \textbf{1-Butanol (CH$_3$CH$_2$CH$_2$CH$_2$OH)}
    \begin{center}
        % Lewis Structure for 1-Butanol
        \chemfig{H-C(-[:30]H)(-[:-30]H)-C(-[:30]H)(-[:-30]H)-C(-[:30]H)(-[:-30]H)-C(-[:30]H)(-[:-30]H)-O-H}
    \end{center}
    - \textbf{VSEPR Shape}: Tetrahedral around each carbon, bent around oxygen.

    \item \textbf{Hexane (CH$_3$(CH$_2$)$_4$CH$_3$)}
    \begin{center}
        % Lewis Structure for Hexane
        \chemfig{H-C(-[:30]H)(-[:-30]H)-C(-[:30]H)(-[:-30]H)-C(-[:30]H)(-[:-30]H)-C(-[:30]H)(-[:-30]H)-C(-[:30]H)(-[:-30]H)-C(-[:30]H)(-[:-30]H)-H}
    \end{center}
    - \textbf{VSEPR Shape}: Tetrahedral around each carbon.

    \item \textbf{Acetone (CH$_3$COCH$_3$)}
    \begin{center}
        % Lewis Structure for Acetone
        \chemfig{H-C(-[:30]H)(-[:-30]H)-C(=[:90]O)-C(-[:30]H)(-[:-30]H)-H}
    \end{center}
    - \textbf{VSEPR Shape}: Trigonal planar around C=O, tetrahedral around other carbons.

    \item \textbf{Water (H$_2$O)}
    \begin{center}
        % Lewis Structure for Water
        \chemfig{H-O-H}
    \end{center}
    - \textbf{VSEPR Shape}: Bent (due to two lone pairs on oxygen).
\end{itemize}
\end{document}