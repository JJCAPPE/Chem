\documentclass{article}

\usepackage{amsmath}
\usepackage{geometry}
\usepackage{booktabs}
\usepackage{array}
\usepackage{longtable}
\usepackage{hyperref}

\geometry{margin=1in}

\begin{document}

\title{Prelab Notes: Measuring Atomic Emission Spectra}
\date{8/10/24 - Giacomo Cappelletto}
\maketitle

\section*{Safety Concerns}

\begin{itemize}
    \item \textbf{High Voltage:} Avoid touching metal connectors on the emission lamps while they are powered. Lamps can become hot during use, so allow them to cool before handling.
    \item \textbf{Spectrometer Use:} Do not point the spectrometer at bright or hazardous light sources, such as the sun, to prevent eye damage.
    \item \textbf{Hygiene:} Wear gloves or wash your hands before and after using the spectrometer, as they are not sanitized between uses.
\end{itemize}

\section*{Part I: Measuring the Atomic Emission Spectra}

\subsection*{Using the Handheld Spectrometer}

\subsubsection*{Setup}

\begin{itemize}
    \item \textbf{Hold} the narrow end (ocular window) close to your eye.
    \item \textbf{Point} the wider end with the incident slit toward the light source.
    \item \textbf{Adjust} distances for a clear view of the emission spectrum on the internal scale.
\end{itemize}

\subsubsection*{Reading Wavelengths}

\begin{itemize}
    \item \textbf{Wavelength scale} is marked every 10 nm; numbers indicate every 100 nm (e.g., ``4'' represents 400 nm).
    \item \textbf{Align} the center of each spectral line with the scale above it.
    \item \textbf{Estimate} wavelengths between markings as accurately as possible.
\end{itemize}

\subsection*{Procedure Steps}

\begin{enumerate}
    \item \textbf{Observing a Non-Atomic Light Source}
    
    \begin{itemize}
        \item \textbf{Action:} Use the spectrometer to observe an everyday light source (e.g., overhead lab lights).
        \item \textbf{Qualitative Description:}
        
        \vspace{1cm}
        \textit{Describe the observed spectrum:}
        
        \vspace{4cm}
        
        \item \textbf{Questions:}
        
        \textit{How does this spectrum differ from atomic emission spectra?}
        
        \vspace{4cm}
    \end{itemize}
    
    \item \textbf{Observing the Hydrogen Emission Spectrum}
    
    \begin{itemize}
        \item \textbf{Action:}
        \begin{itemize}
            \item Turn on the hydrogen lamp.
            \item Use the spectrometer to observe hydrogen's emission lines.
            \item Shield the spectrometer from other light sources if needed.
        \end{itemize}
        \item \textbf{Ensure:}
        \begin{itemize}
            \item Only hydrogen spectral lines are observed (ask to turn off overhead lights if necessary).
        \end{itemize}
        \item \textbf{Data Recording:}
        
        \textit{Record the color and measured wavelength of each visible emission line.}
        
        \begin{center}
        \renewcommand{\arraystretch}{1.5}
        \begin{tabular}{|p{3cm}|p{4cm}|p{6cm}|}
        \hline
        \textbf{Color} & \textbf{Measured Wavelength (nm)} & \textbf{Observations} \\ \hline
        & & \\ \hline
        & & \\ \hline
        & & \\ \hline
        \end{tabular}
        \end{center}
        
        \textit{Note any variations in brightness or width of the lines.}
        
    \end{itemize}
    
    \item \textbf{Group Discussion and Data Comparison}
    
    \begin{itemize}
        \item \textbf{Action:}
        \begin{itemize}
            \item Compare your observed wavelengths with group members.
            \item Discuss any discrepancies and possible reasons.
        \end{itemize}
        \item \textbf{Data Table:}
        
        \begin{center}
        \renewcommand{\arraystretch}{1.5}
        \begin{tabular}{|p{3cm}|p{3cm}|p{3cm}|p{3cm}|p{3cm}|}
        \hline
        \textbf{Color} & \textbf{Your Measurement (nm)} & \textbf{Member 1 (nm)} & \textbf{Member 2 (nm)} & \textbf{Average Wavelength (nm)} \\ \hline
        & & & & \\ \hline
        & & & & \\ \hline
        & & & & \\ \hline
        \end{tabular}
        \end{center}
        
        \textit{Calculate the average wavelength for each line.}
    \end{itemize}
    
    \item \textbf{Observing Another Atomic Emission Lamp}
    
    \begin{itemize}
        \item \textbf{Action:}
        \begin{itemize}
            \item Choose a second atomic lamp (e.g., helium, neon).
            \item Repeat steps similar to those for hydrogen.
        \end{itemize}
        \item \textbf{Element Observed:} \underline{\hspace{5cm}}
        \item \textbf{Data Recording:}
        
        \begin{center}
        \renewcommand{\arraystretch}{1.5}
        \begin{tabular}{|p{3cm}|p{4cm}|p{6cm}|}
        \hline
        \textbf{Color} & \textbf{Measured Wavelength (nm)} & \textbf{Observations} \\ \hline
        & & \\ \hline
        & & \\ \hline
        & & \\ \hline
        \end{tabular}
        \end{center}
        
        \item \textbf{Group Data Comparison:}
        
        \begin{center}
        \renewcommand{\arraystretch}{1.5}
        \begin{tabular}{|p{3cm}|p{3cm}|p{3cm}|p{3cm}|p{3cm}|}
        \hline
        \textbf{Color} & \textbf{Your Measurement (nm)} & \textbf{Member 1 (nm)} & \textbf{Member 2 (nm)} & \textbf{Average Wavelength (nm)} \\ \hline
        & & & & \\ \hline
        & & & & \\ \hline
        & & & & \\ \hline
        \end{tabular}
        \end{center}
    \end{itemize}
    
    \item \textbf{Final Steps}
    
    \begin{itemize}
        \item \textbf{Action:}
        \begin{itemize}
            \item Turn off all emission lamps when finished.
            \item Return the spectrometer to the instructor.
            \item Obtain the post-lab assignment.
        \end{itemize}
    \end{itemize}
    
\end{enumerate}

\section*{Expected Wavelengths for Hydrogen Balmer Series}

\textbf{Transitions where \( n_f = 2 \) and \( n_i > 2 \):}

\begin{center}
\renewcommand{\arraystretch}{1.5}
\begin{tabular}{|c|c|}
\hline
\textbf{Transition (\( n_i \rightarrow n_f \))} & \textbf{Expected Wavelength (nm)} \\ \hline
\( 3 \rightarrow 2 \) & 656.3 \\ \hline
\( 4 \rightarrow 2 \) & 486.1 \\ \hline
\( 5 \rightarrow 2 \) & 434.0 \\ \hline
\( 6 \rightarrow 2 \) & 410.2 \\ \hline
\end{tabular}
\end{center}

\textit{Use these values to help identify the observed spectral lines.}

\section*{Space for Calculations and Additional Observations}

\subsection*{Calculations}

\vspace{6cm}

\subsection*{Additional Observations}

\vspace{6cm}

\noindent\textbf{Note:} Ensure all measurements include proper units and significant figures based on the spectrometer's scale markings.

\end{document}